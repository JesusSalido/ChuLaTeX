\documentclass[10pt,landscape,a4paper]{article}
%\setmargins{6mm}{6mm}{6mm}{6mm}
\usepackage{multicol}
\usepackage{calc}
\usepackage{ifthen}
\usepackage[landscape]{geometry}
% opciones para el castellano
\usepackage[utf8]{inputenx}
\usepackage{pifont}
\usepackage[T1]{fontenc}
\usepackage{type1ec}
\usepackage[spanish,es-sloppy]{babel}
\usepackage{newtxtext}
%\usepackage{mathptmx}
\usepackage[scaled=0.85]{couriers}
\usepackage{textcomp,marvosym}
\usepackage{graphicx}
\usepackage{rotating}
\usepackage{xcolor}
\definecolor{palered}{rgb}{0.78, 0.03, 0.08}
\usepackage[pdftex,
	colorlinks=true, 
	anchorcolor=palered, 
	citecolor=palered, 
	filecolor=palered, 
	urlcolor=palered,
	linkcolor=palered]{hyperref}

\ifthenelse{\lengthtest { \paperwidth = 11in}}
	{ \geometry{top=.4in,left=.4in,right=.4in,bottom=.4in} }
	{\ifthenelse{ \lengthtest{ \paperwidth = 297mm}}
		{\geometry{top=0.6cm,left=0.6cm,right=0.6cm,bottom=0.6cm} }
		{\geometry{top=0.6cm,left=0.6cm,right=0.6cm,bottom=0.6cm} }
	}

% Turn off header and footer
\pagestyle{empty}
 

% Redefine section commands to use less space
\makeatletter
\renewcommand{\section}{\@startsection{section}{1}{0mm}%
                                {-1ex plus -.5ex minus -.2ex}%
                                {0.5ex plus .2ex}%x
                                {\normalfont\large\bfseries}}
\renewcommand{\subsection}{\@startsection{subsection}{2}{0mm}%
                                {-1explus -.5ex minus -.2ex}%
                                {0.5ex plus .2ex}%
                                {\normalfont\normalsize\bfseries}}
\renewcommand{\subsubsection}{\@startsection{subsubsection}{3}{0mm}%
                                {-1ex plus -.5ex minus -.2ex}%
                                {1ex plus .2ex}%
                                {\normalfont\small\bfseries}}
\makeatother

% Define BibTeX command
\def\BibTeX{{\rm B\kern-.05em{\sc i\kern-.025em b}\kern-.08em
    T\kern-.1667em\lower.7ex\hbox{E}\kern-.125emX}}

% Don't print section numbers
\setcounter{secnumdepth}{0}


\setlength{\parindent}{0pt}
\setlength{\parskip}{0pt plus 0.5ex}


% -----------------------------------------------------------------------

\begin{document}

\raggedright
%\scriptsize
\footnotesize
\begin{multicols}{3}


% multicol parameters
% These lengths are set only within the two main columns
%\setlength{\columnseprule}{0.25pt}
\setlength{\premulticols}{1pt}
\setlength{\postmulticols}{1pt}
\setlength{\multicolsep}{1pt}
\setlength{\columnsep}{2pt}

%\begin{flushleft}
%     \Large{\textbf{Chu\LaTeX}} \\
%     {Una <<chuleta>> muy <<chula>> de \LaTeX} \\
%\end{flushleft}
\begin{flushleft}
     {\Large\textbf{Chu\LaTeX:}} {\large Una <<chuleta>> muy <<chula>> de \LaTeX} \\
     \href{http://visilab.etsii.uclm.es/?page_id=1468}{\large \LaTeX{} esencial para TFG (ESI-UCLM)}
\end{flushleft}

%-------------------------
\section{Clases de documento} 
\ding{118}  Uso: \verb!\documentclass[!\emph{opcs.}\verb!]{!\emph{<clase>}\verb!}!
\begin{tabular}{@{}ll@{}}
%\emph{Clases}         & (Ajustes predefinidos)\\
\verb!book!    & A doble cara por defecto, sin \verb+\abstract+ \\
\verb!report!  & Igual a \texttt{book}, con \verb+\abstract+ y sin \verb!\part! \\
\verb!article! & Sin \verb!\part! ni \verb!\chapter! \\[0.7mm]
%\verb!letter!  & Carta. \\
%\verb!slides!  & Transparencias para presentaciones.\\[0.7mm]
\end{tabular}

\newlength{\MyLen}
\settowidth{\MyLen}{\texttt{letterpaper}/\texttt{a4paper} \ }
\begin{tabular}{@{}ll@{}}
                %\hline
\emph{Opciones} & (Modifican valores predefinidos) \\
\texttt{10pt}/\texttt{11pt}/\texttt{12pt} & Tamaño de letra \\
\texttt{a4paper} & Tamaño del papel \\
\texttt{twocolumn} & Texto a dos columnas \\
\texttt{twoside}   & Ajustes para imprimir en ambas caras del papel \\
\texttt{openany}/\texttt{openright} & Inicia cap. en cualquier página o sólo a la dcha.
\end{tabular}





%-------------------------
\subsection{Paquetes imprescindibles}
\ding{118}  Uso: \verb!\usepackage[!\emph{opcs.}\verb!]{!\emph{<paquete>}\verb!}!\\[0.7mm]


\settowidth{\MyLen}{\texttt{.multicol.} }
\begin{tabular}{@{}p{\the\MyLen}%
                @{}p{\linewidth-\the\MyLen}@{}}
\texttt{inputenx}	& Codificación del texto de entrada [\texttt{utf8})]\\
\texttt{babel}		& Soporte multilingue [\texttt{spanish}, \texttt{english}]\\
\texttt{geometry}	& Definición de geometría de la página (tamaño y márgenes)\\
\texttt{textcomp}	& Caracteres especiales (junto a paquetes \texttt{marvosym} y \texttt{pifont}). \\
\texttt{ams}$\langle ccc \rangle$ & Inc. de elementos matemáticos (con $ccc=$\texttt{math}, \texttt{thm}, \texttt{symb}, \texttt{fonts})\\
\texttt{fontenc}	& Codificación de salida [\texttt{T1}]\\
%\texttt{setspace}	& Esp. interlineal (Ops.:\texttt{{single}, \texttt{double}, \texttt{onehalf}}\texttt{spacing}). \\
\texttt{url}		& Formato adecuado de URLs (mediante \emph{cmd.} \verb!\url!). \\
\texttt{hyperref}	& Creación de PDF enriquecidos\\
\texttt{graphicx}	& Inclusión de figuras\\
\texttt{rotating}	& Rotación de objetos (colocación apaisada de fíguras y tablas)\\
\texttt{caption}	& Configuración de títulos de figuras y tablas \\
%\texttt{secsty}	& Configuración de títulos de secciones \\
\texttt{fancyhdr}	& Facilita la \emph{conf}. nativa de cabeceras y pies de página.\\
\texttt{parskip}	& Control de espaciado entre párrafos\\
%\texttt{paralist}	& Mejora el control de listas\\
\texttt{multicol}	& Listas en varias columnas\\
\texttt{listings}	& Permite insertar porciones de código fuente\\
%\texttt{pdfpages}	& Permite insertar páginas de un fichero PDF\\
%\texttt{etex}		& Soluciona errores de desbordamiento\\
\end{tabular}



%-------------------------
\section{Estructura del documento}
\settowidth{\MyLen}{\texttt{.lstlistoflistings.} }
\begin{tabular}{@{}p{\the\MyLen}%
                @{}p{\linewidth-\the\MyLen}@{}}
\verb!\title{!\emph{texto}\verb!}!	& Título del documento\\
\verb!\author{!\emph{texto}\verb!}!	& Autor del documento\\
\verb!\date{!\emph{texto}\verb!}!	& Fecha (\verb+\today+, genera fecha actual) \\
\verb!\begin{document}! 			& Declaración de inicio tras el preámbulo\\
\verb!\maketitle! 					& Genera pág. de título con info\ suministrada. \\
\verb!\tableofcontents! 			& Crea el índice de contenidos (TOC)\\
\verb!\listoffigures!   			& Lista de figuras \\
\verb!\listoftables!    			& Lista de cuadros (tablas) \\
\verb!\lstlistoflistings!    		& Lista de listados \\
\verb!\begin{abstract}! 			& Resumen del documento \\[0.7mm] 
\end{tabular}




%-------------------------
\subsection{Pies y cabeceras}
\verb!\pagestyle{!\emph{sty.}\verb!}!: Estilo (\texttt{empty}, \texttt{plain}, \texttt{headings} y \texttt{myheadings})\\
\verb!\thispagestyle{!\emph{sty.}\verb!}!: Estilo sólo aplicado a pág. actual\\




%-------------------------
\subsection{Secciones del documento y nivel jerárquico}
\begin{multicols}{2}
(-1) \verb!\part[!\emph{tít.corto$^*$}\verb!]{!\emph{título}\verb!}!  \\
(0) \verb!\chapter[!\emph{tít.corto}\verb!]{!\emph{título}\verb!}!  \\
(1) \verb!\section[!\emph{tít.corto}\verb!]{!\emph{título}\verb!}!  \\
(2) \verb!\subsection[!\emph{tít.corto}\verb!]{!\emph{título}\verb!}!  \\
(3) \verb!\subsubsection[!\emph{tít.corto}\verb!]{!\emph{título}\verb!}!  \\
\end{multicols}


$^*$Cuando se suministra, un \emph{título corto} es el que aparece en el (TOC) \\[0.7mm]
\verb!\section*{!\emph{título}\verb!}!: sección sin numerar, ni incluida en TOC\\
\verb!\setcounter{secnumdepth}{!$x$\verb!}!: suprime la numeración al nivel $>x$
\verb!\setcounter{tocdepth}{!$x$\verb!}!: suprime del TOC las secciones de nivel $>x$
\verb!\the!$\langle cnt \rangle$: imprime el valor del contador $cnt$ (el valor de $cnt$ puede ser \texttt{page}, \texttt{chapter}, \texttt{section}, etc.)






%-------------------------
%-------------------------
\subsection{Referencias cruzadas y notas al pie}
\settowidth{\MyLen}{\texttt{.pageref.marker..}}
\begin{tabular}{@{}p{\the\MyLen}%
                @{}p{\linewidth-\the\MyLen}@{}}
\verb!\label{!\emph{id.label}\verb!}!   &  Marca (p.ej. \texttt{sec:item}) para una referencia \\
\verb!\ref{!\emph{id.label}\verb!}!   &  Genera el nº de sección \\
\verb!\pageref{!\emph{id.label}\verb!}! &  Genera el nº de página \\
\verb!\autoref{!\emph{id.label}\verb!}! &  Con paquete \texttt{hyperref} genera sec. y numeración \\
\verb!\footnote{!\emph{nota}\verb!}!  &  Genera una nota al pie de página \\
\end{tabular}





%-------------------------
\subsection{Listas}
\settowidth{\MyLen}{\texttt{.begin.dingautolist.....}}
\begin{tabular}{@{}p{\the\MyLen}%
                @{}p{\linewidth-\the\MyLen}@{}}
\verb!\begin{enumerate}!        &  Enumeración \\
%\verb!\begin{compactenum}!      &  Enumeración compacta (paquete \texttt{paralist})\\
\verb!\begin{dingautolist}{!$n$\verb!}!    &  Enumeración desde símbolo \texttt{dingbat} $n$\\
\verb!\begin{itemize}!          &  Lista marcada \\
%\verb!\begin{compactitem}!      &  Lista marcada compacta (paquete \texttt{paralist}) \\
\verb!\begin{dinglist}{!$n$\verb!}!   &  Lista marcada con símbolo $n$ \texttt{dingbat}.\\
\verb!\begin{description}!      &  Lista descriptiva. \\
\verb!\begin{multicol}{<n>}!    &  Lista en $n$ columnas (paquete \texttt{multicol}) \\[0.5mm]
\end{tabular}

\verb!\item[!\emph{c}\verb!]! \emph{texto}: Entrada de lista usando \emph{c} como marca (obligado en \emph{descrip.})\\
%\verb!\begin{quote}!: Bloque sangrado para una cita.\\
%\verb!\begin{quotation}!: Como quote con párrafos sangrados.\\
%\verb!\begin{verse}!: Para versos (\verb!\\! separando versos de una estrofa).\\




%-------------------------
\section{Atributos del texto}


%-------------------------
% \subsection{Estilos}
\newcommand{\FontCmd}[3]{\PBS\verb!\#1{!\emph{texto}\verb!}!  \> %
                         \verb!{\#2 !\emph{texto}\verb!}! \> %
                         \#1{#3}}
\begin{tabular}{@{}l@{}l@{}l@{}}
\emph{Orden (Pref.)}  & \emph{Declaración} & \emph{Efecto} \\ \hline
\verb!\textrm{!\emph{texto}\verb!}!                    & %
        \verb!{\rmfamily !\emph{texto}\verb!}    !               & %
        \textrm{Tipo redondo (Serif)} \\
\verb!\textsf{!\emph{texto}\verb!}!                    & %
        \verb!{\sffamily !\emph{texto}\verb!}!               & %
        \textsf{Palo seco (Sans Serif)} \\
\verb!\texttt{!\emph{texto}\verb!}!                    & %
        \verb!{\ttfamily !\emph{texto}\verb!}!               & %
        \texttt{Teletipo (monoespaciada)} \\
\verb!\textmd{!\emph{texto}\verb!}!                    & %
        \verb!{\mdseries !\emph{texto}\verb!}!               & %
        \textmd{Normal} \\
\verb!\textbf{!\emph{texto}\verb!}!                    & %
        \verb!{\bfseries !\emph{texto}\verb!}!               & %
        \textbf{Negrita} \\
\verb!\textup{!\emph{texto}\verb!}!                    & %
        \verb!{\upshape !\emph{texto}\verb!}!               & %
        \textup{Vertical} \\
\verb!\textit{!\textit{texto}\verb!}!                    & %
        \verb!{\itshape !\emph{texto}\verb!}!               & %
        \emph{Itálica} \\
\verb!\textsl{!\emph{texto}\verb!}!                    & %
        \verb!{\slshape !\emph{texto}\verb!}!               & %
        \textsl{Inclinada} \\
\verb!\textsc{!\emph{texto}\verb!}!                    & %
        \verb!{\scshape !\emph{texto}\verb!}!               & %
        \textsc{Versalitas} \\
\verb!\emph{!\emph{texto}\verb!}!                      & %
        \verb!{\em !\emph{texto}\verb!}!               & %
        \emph{Enfatizado (cursiva)} \\
\verb!\textnormal{!\emph{texto}\verb!}  !                & %
        \verb!{\normalfont !\emph{texto}\verb!}!       & %
        \textnormal{Predefinida} \\[0.7mm]
\end{tabular}

%Usar preferentemente la \emph{orden} frente a la \emph{declaración} (mejor ajuste de espacio) y el texto  \emph{enfatizado} frente a la \textit{itálica} (adaptación automática de tipo).\\[0.7mm]



%-------------------------
\textbf{Texto literal}\\
\settowidth{\MyLen}{\texttt{.begin.verbatim..} }
\begin{tabular}{@{}p{\the\MyLen}%
                @{}p{\linewidth-\the\MyLen}@{}}
\verb@\begin{verbatim}@ & Entorno literal \\
\verb@\begin{verbatim*}@ & Muestra los espacios como en \verb*@ @ \\
\verb@\verb|text|@ & El texto entre los delimitadores es literal
%                   &(el delimitador puede ser: `\texttt{|}', `\texttt{+}', `\texttt{@}', ...).
\end{tabular}





%-------------------------
\subsection{Justificación y ubicación de párrafos}

\begin{tabular}{@{}lll@{}}
\emph{Entorno}  &  \emph{Declaración}  & \emph{Efecto}\\ \hline
\verb!\begin{center}!      & \verb!\centering!  & \makebox[15ex][c]{centrado} \\
\verb!\begin{flushleft}!  & \verb!\raggedright! & \makebox[15ex][r]{just. dcha.}\\
\verb!\begin{flushright}! & \verb!\raggedleft!  & \makebox[15ex][l]{just. izda.}\\[0.7mm]
\end{tabular}

\textbf{Variables de longitud}\\
\verb!\parindent!: Longitud de sangrado 1ª línea de párrafo (15pt, predef.) \\
\verb!\parskip!: Espaciado entre párrafos (0pt plus 1pt, predef.) \\
\verb!\textwidth!: Ancho del texto \\
\verb!\linewidth!: Ancho de la línea en el entorno local \\





%-------------------------
\subsection{Tamaño de letra respecto al texto [predefinido]}
\begin{tabular}{@{}r@{\hspace{0.5ex}}r@{\hspace{0.5ex}}r@{\hspace{0.5ex}}r|@{\hspace{0.5ex}}r@{\hspace{0.5ex}}r@{\hspace{0.5ex}}r@{\hspace{0.5ex}}r@{}}
\emph{Declaración}& \textbf{[10pt} & \textbf{11pt} & \textbf{12pt]} &    \emph{Declaración}& \textbf{[10pt} & \textbf{11pt} & \textbf{12pt]}\\
\verb!\tiny!&5pt&6pt&6pt&\verb!\large!&12pt&12pt&14pt\\
\verb!\scriptsize!&7pt&8pt&8pt&\verb!\Large!&14pt&14pt&17pt\\
\verb!\footnotesize!&8pt&9pt&10pt&\verb!\LARGE!&17pt&17pt&20pt\\
\verb!\small!&9pt&10pt&11pt&\verb!\huge!&20pt&20pt&25pt\\
\verb!\normalsize!&10pt&11pt&12pt&\verb!\Huge!&25pt&25pt&25pt\\[0.7mm]
\end{tabular}

Usados dentro de llaves o afectando a todo un entorno (o documento)

\vspace{1cm}





%-------------------------
%-------------------------
\subsection{Tipografías en \LaTeX}

\ding{118} En español debe emplearse paquete \texttt{fontenc} con codificación \texttt{T1}\\
(en inglés no es preciso)
\begin{tabular}{@{}c|@{\hspace{1ex}}c@{\hspace{1ex}}c@{\hspace{1ex}}c@{}}
\textbf{Paquete}   &  \textbf{Romama} (rm) & \textbf{Paloseco} (sf) & \textbf{Teletipo} (tt) \\ \hline
(predefinida) &  {\usefont{T1}{cmr}{m}{n} CM Roman} & {\usefont{T1}{cmss}{m}{n}CMSSerif} & {\usefont{T1}{cmtt}{m}{n}CMttype} \\
\texttt{lmodern}   &  {\usefont{T1}{lmr}{m}{n} LM Roman} & {\usefont{T1}{lmss}{m}{n}LMSSerif} & {\usefont{T1}{lmtt}{m}{n}LMttype} \\
\texttt{mathpazo}  & {\usefont{T1}{ppl}{m}{n}Palatino} &              & \\
\texttt{mathptmx}  & {\usefont{T1}{ptm}{m}{n}Times}   &              & \\
\texttt{helvet}    &  &  {\usefont{T1}{phv}{m}{n}Helvetica}   & \\
\texttt{avant}     &  &  {\usefont{T1}{pag}{m}{n}Avant Garde} & \\
\texttt{courier}   &  &  & {\usefont{T1}{pcr}{m}{n}Courier}\\[0.7mm]
\end{tabular}

\ding{118} Cuando una tipografía no posee la versión de alguna de las familias se\\
emplea la familia correspondiente de la tipografía predefinida.\\
Otros paquetes:\\

\texttt{[opcs.]\textbf{libertine}}: Libertine\\
\texttt{[opcs.]\textbf{newtxtext}} + \texttt{\textbf{newtxmath}}: Times\\
\texttt{[opcs.]\textbf{newpxtext}} + \texttt{\textbf{newpxmath}}: Palatino\\ 
Opciones: \texttt{scaled} (escalado), \texttt{osf} (\emph{old style figures})



%-------------------------
\subsection{Saltos de línea, página y espacios}
\settowidth{\MyLen}{\texttt{.rule.w..h...} }
\begin{tabular}{@{}p{\the\MyLen}%
                @{}p{\linewidth-\the\MyLen}@{}}
%\verb!\\!          &  Salto de línea sin iniciar otro párrafo.  \\
\verb!\newline!			&  Salto de línea sin iniciar párrafo  \\
\verb!\\[!$l$\verb!]!	&  Salto de línea (opc. long. $l$) en el mismo párrafo   \\
\verb!\\*!         		&  Inhibe salto de página posterior \\
\verb!\noindent!   		&  No sangra la línea actual \\
\verb!\newpage!    		&  Salto de página \\
\verb!\pagebreak!  		&  Completa página y fuerza nuevo inicio \\
\verb!\kill!       		&  Inhibe impresión de línea actual \\
\verb!~!       			&  Espacio de no separación (p.ej. \verb!D.E.~Knuth!) \\
\verb!$\sim$!  			&  Imprime $\sim$ en vez de \~{} (producido por \verb!\~{}!) \\
\verb!\hspace{!$l$\verb!}! 		& Espacio horizontal de medida $l$
                                (p.ej. $l=\mathtt{20pt}$) \\
\verb!\hfill!  			& Relleno en blanco hasta completar línea\\
\verb!\hrulefill!  		& Relleno con raya \\
\verb!\dotfill!    		& Relleno con puntos \\
\verb!\dingfill{!$c$\verb!}!    & Relleno con símbolo $c$ de tabla \texttt{dingbat} \\
\verb!\dingline{!$c$\verb!}!    & Línea formada por símbolo $c$ de tabla \texttt{dingbat} \\
\verb!\rule{!$w$\verb!}{!$h$\verb!}! & Línea de ancho $w$ y altura $h$ \\
\verb!\vspace{!$l$\verb!}! 		& Espacio vertical de medida $l$ \\
\verb!\vfill!  			& Espacio de relleno vertical hasta completar página \\
\end{tabular}





%-------------------------
\subsection{División de palabras y párrafos}
%La división automática de palabras (guionado) está activada y adaptada al idioma del texto mediante la opción de paquete \emph{babel} (recordar configurar la distribución).\\[0.7mm]
\verb!\hyphenation{pa-la-bra}!: Guionado para \emph{palabra} (en preámbulo)\\
\verb!--!: Guión medio (p.ej. rango 1--5) (sólo en idioma inglés)\\
\verb!---!: Guión largo o parentético (p.ej. ---dijo él---)\\
\verb!~-! y \verb!~---!: Guiones no separables (disponibles con \texttt{[spanish]babel})\\[0.7mm]

%\verb!`'!: `Comilla simple'; \verb!``''!: ``Comilla doble inglesa''; \verb!<<>>!: <<
%Comilla doble>>.\\[0.7mm]

\textbf{Desactivación de división} (aumentando \texttt{n}, hasta 10000)\\
\verb!\hypenpenalty=n \exhyphenpenalty=n \sloppy!\\

\verb!\widowpenalty=n \clubpenalty=n! (Evita viudas y huérfanas) \\

\verb!\interfootnotelinepenalty=n! (Evita división de notas en varias págs.)

%\begin{tabular}{@{}llll@{}}
%\emph{Nombre} & \emph{Código} & \emph{Ejemplo} & \emph{Uso} \\
%guión  & \verb!-!   & -          & En palabras \\
%guión medio & \verb!--!  & 1--5           & Intervalos entre números* \\
%guión largo & \verb!---! & Sí ---dijo él---    & Guiones largos (parentéticos)\\
%comilla & \verb!` '! & `c' & Comilla simple\\
%comilla doble &\verb!`` ''! & ``see'' & Comilla doble*\\
%comilla doble &\verb!<< >>! & <<ojo>> & Comilla doble española\\[0.7mm]
%\end{tabular}
%
%*Sólo se debe emplear en tipografía anglosajona, nunca en española.







%-------------------------
\subsection{Símbolos textuales {\small (paquetes \texttt{textcomp} y \texttt{marvosym})}}
\begin{tabular}{@{}l@{\hspace{1em}}l@{\hspace{2em}}l@{\hspace{1em}}l@{\hspace{2em}}l@{\hspace{1em}}l@{\hspace{2em}}l@{\hspace{1em}}l@{}}
\&              &  \verb!\&! &
\_              &  \verb!\_! &
\ldots          &  \verb!\dots! &
\textbullet     &  \verb!\textbullet! \\
\$              &  \verb!\$! &
\^{}            &  \verb!\^{}! &
\textbar        &  \verb!\textbar! &
\textbackslash  &  \verb!\textbackslash! \\
\%              &  \verb!\%! &
\~{}            &  \verb!\~{}! &
\#              &  \verb!\#! &
\S              &  \verb!\S! \\
\P              &  \verb!\P! &
\dag            &  \verb!\dag! &
\ddag           &  \verb!\ddag! &
\textperiodcentered              &  \verb!\textperiodcentered!
\end{tabular}

\begin{tabular}{@{}l@{\hspace{1em}}l@{\hspace{2em}}l@{\hspace{1em}}l@{\hspace{2em}}}
\textregistered	&  \verb!\textregistered!	&
\EUR 			&  \verb!\EUR!  (paquete \texttt{marvosym}) \\
\copyright		&  \verb!\copyright! 		& 
\texteuro		&  \verb!\texteuro!  (paquete \texttt{textcomp}) \\
\TeX			& \verb!\TeX!				&
\texttrademark  &  \verb!\texttrademark! \\
\LaTeX 			&  \verb!\LaTeX! 			& \LaTeXe	&	\verb!\LaTeXe!\\
\end{tabular}









%-------------------------
%-------------------------
\section{Símbolos dingbat {\small (paquete \texttt{pifont})}}

\verb!\ding{n}!: Genera símbolo de tabla \emph{dinbat}\\

\newlength{\ancho}
\settowidth{\ancho}{999-s-}

\begin{tabular}{@{}l@{}l@{}l@{}l@{}l@{}l@{}l@{}l@{}l@{}l@{}}
33  \ding{33} &  34 \ding{34} &  35 \ding{35} &  36 \ding{36} &  37 \ding{37} &
38  \ding{38} &  39 \ding{39} &  40 \ding{40} &  41 \ding{41} &  42 \ding{42} \\
43  \ding{43} &  44 \ding{44} &  45 \ding{45} &  46 \ding{46} &  47 \ding{47} & 
48  \ding{48} &  49 \ding{49} &  50 \ding{50} &  51 \ding{51} &  52 \ding{52} \\
53  \ding{53} &  54 \ding{54} &  55 \ding{55} &  56 \ding{56} &  57 \ding{57} &
58  \ding{58} &  59 \ding{59} &  60 \ding{60} &  61 \ding{61} &  62 \ding{62} \\
%53  \ding{53} &  54 \ding{54} &  55 \ding{55} &  56 \ding{56} &  57 \ding{57} &
%58  \ding{58} &  59 \ding{59} &  60 \ding{60} &  61 \ding{61} &  62 \ding{62} \\
63  \ding{63} &  64 \ding{64} &  65 \ding{65} &  66 \ding{66} &  67 \ding{67} &
68  \ding{68} &  69 \ding{69} &  70 \ding{70} &  71 \ding{71} &  72 \ding{72} \\
73  \ding{73} &  74 \ding{74} &  75 \ding{75} &  76 \ding{76} &  77 \ding{77} &
78  \ding{78} &  79 \ding{79} &  80 \ding{80} &  81 \ding{81} &  82 \ding{82} \\
83  \ding{83} &  84 \ding{84} &  85 \ding{85} &  86 \ding{86} &  87 \ding{87} &
88  \ding{88} &  89 \ding{89} &  90 \ding{90} &  91 \ding{91} &  92 \ding{92} \\
93  \ding{93} &  94 \ding{94} &  95 \ding{95} &  96 \ding{96} &  97 \ding{97} &
98  \ding{98} &  99 \ding{99} & 100 \ding{100}& 101 \ding{101}& 102 \ding{102}\\
103 \ding{103}& 104 \ding{104}& 105 \ding{105}& 106 \ding{106}& 107 \ding{107}&
108 \ding{108}& 109 \ding{109}& 110 \ding{110}& 111 \ding{111}& 112 \ding{112}\\
113 \ding{113}& 114 \ding{114}& 115 \ding{115}& 116 \ding{116}& 117 \ding{117}&
118 \ding{118}& 119 \ding{119}& 120 \ding{120}& 121 \ding{121}& 122 \ding{122}\\
123 \ding{123}& 124 \ding{124}& 125 \ding{125}& 126 \ding{126}&               &
              &               &               & 161 \ding{161}& 162 \ding{162}\\
163 \ding{163}& 164 \ding{164}& 165 \ding{165}& 166 \ding{166}& 167 \ding{167}&
168 \ding{168}& 169 \ding{169}& 170 \ding{170}& 171 \ding{171}& 172 \ding{172}\\
173 \ding{173}& 174 \ding{174}& 175 \ding{175}& 176 \ding{176}& 177 \ding{177}&
178 \ding{178}& 179 \ding{179}& 180 \ding{180}& 181 \ding{181}& 182 \ding{182}\\
183 \ding{183}& 184 \ding{184}& 185 \ding{185}& 186 \ding{186}& 187 \ding{187}&
188 \ding{188}& 189 \ding{189}& 190 \ding{190}& 191 \ding{191}& 192 \ding{192}\\
193 \ding{193}& 194 \ding{194}& 195 \ding{195}& 196 \ding{196}& 197 \ding{197}&
198 \ding{198}& 199 \ding{199}& 200 \ding{200}& 201 \ding{201}& 202 \ding{202}\\
203 \ding{203}& 204 \ding{204}& 205 \ding{205}& 206 \ding{206}& 207 \ding{207}&
208 \ding{208}& 209 \ding{209}& 210 \ding{210}& 211 \ding{211}& 212 \ding{212}\\
213 \ding{213}& 214 \ding{214}& 215 \ding{215}& 216 \ding{216}& 217 \ding{217}&
218 \ding{218}& 219 \ding{219}& 220 \ding{220}& 221 \ding{221}& 222 \ding{222}\\
223 \ding{223}& 224 \ding{224}& 225 \ding{225}& 226 \ding{226}& 227 \ding{227}&
228 \ding{228}& 229 \ding{229}& 230 \ding{230}& 231 \ding{231}& 232 \ding{232}\\
233 \ding{233}& 234 \ding{234}& 235 \ding{235}& 236 \ding{236}& 237 \ding{237}&
238 \ding{238}& 239 \ding{239}&               & 241 \ding{241}& 242 \ding{242}\\
243 \ding{243}& 244 \ding{244}& 245 \ding{245}& 246 \ding{246}& 247 \ding{247}&
248 \ding{248}& 249 \ding{249}& 250 \ding{250}& 251 \ding{251}& 252 \ding{252}\\
253 \ding{253}& 254 \ding{254}&               &               &               &
              &               &               &               &               \\
\end{tabular}







%-------------------------
\section{Modo matemático {\small (paquetes \texttt{amsmath}, \texttt{amsthm} y \texttt{amssymb})}} 
\textbf{Modos matemáticos}\\

\ding{172} Ec. condensada en línea: \verb!\(!\emph{ec.}\verb!\)! o \verb!$!\emph{ec.}\verb!$!\\
\ding{173} Ec. sin numerar: \verb!\begin{displaymath}! o \verb!\[!\emph{ec.}\verb!\]! o \verb!$$!\emph{ec.}\verb!$$!\\
\ding{174} Ec. numerada: \verb!\begin{equation}!\\[0.7mm]

\verb!\displaystyle!: ajusta a estilo expandido\\[0.7mm]

\verb!\begin{eqnarray}!: muestra varias fórmulas en sucesivas líneas (numeradas)\\
\verb!\nonumber!: elimina nº de \emph{ec.} en la línea que aparece\\
\verb!\begin{eqnarray*}!: permite fórmulas en varias líneas (sin nº)\\
\verb!\begin{array}!: muestra fórmula extendida a varias líneas (sin nº),\\
con \texttt{\&} y \verb!\\! como en tabla\\[0.7mm]


\textbf{Expresiones comunes}
\begin{tabular}{@{}l@{\hspace{1em}}l@{\hspace{2em}}l@{\hspace{1em}}l@{}}
$y^{x}_{z}$			&  \verb!y^{x}_{z}!    			&
$\lim \limits_{x \to \infty } f(x)$        			&  \verb!\lim\limits_{x\to\infty}f(x)!\\  
$\frac{x}{y}$       &  \verb!\frac{x}{y}!      		&  
$\sum_{k=1}^n$      &  \verb!\sum_{k=1}^n!     			\\  
$\sqrt[n]{x}$       &  \verb!\sqrt[n]{x}!     		&  
$\prod_{k=1}^n$     &  \verb!\prod_{k=1}^n!    		    \\
$n+1\choose k$      &  \verb!n+1\choose k!      	&  
$\int_0^\infty {x^2}$&  \verb!\int_0^\infty {x^2}!    \\
\end{tabular}






%-------------------------
\subsection{Símbolos frecuentes en modo matemático}

\begin{tabular}{@{}l@{\hspace{1ex}}l@{\hspace{1em}}l@{\hspace{1ex}}l@{\hspace{1em}}l@{\hspace{1ex}} l@{\hspace{1em}}l@{\hspace{1ex}}l@{}}
$\leq$          &  \verb!\leq!  	&
$\geq$          &  \verb!\geq!  	&
$\neq$          &  \verb!\neq!  	&
$\approx$       &  \verb!\approx!  	\\
$\times$        &  \verb!\times!  	&
$\div$          &  \verb!\div!  	&
$\pm$           & \verb!\pm!  		&
$\cdot$         &  \verb!\cdot!  	\\
$^{\circ}$      & \verb!^{\circ}! 	&
$\circ$         &  \verb!\circ!  	&
$\prime$        & \verb!\prime!  	&
%$\cdots$        &  \verb!\cdots!  	\\
$\oplus$        & \verb!\oplus!  	\\
$\infty$        & \verb!\infty!  	&
$\neg$          & \verb!\neg!  		&
$\wedge$        & \verb!\wedge!  	&
$\vee$          & \verb!\vee!  		\\
$\supset$       & \verb!\supset!  	&
$\forall$       & \verb!\forall!  	&
$\in$           & \verb!\in!  		&
$\rightarrow$   &  \verb!\rightarrow! \\
$\subset$       & \verb!\subset!  	&
$\exists$       & \verb!\exists!  	&
$\notin$        & \verb!\notin!  	&
$\Rightarrow$   &  \verb!\Rightarrow! \\
$\cup$          & \verb!\cup!  		&
$\cap$          & \verb!\cap!  		&
$\mid$          & \verb!\mid!  		&
$\Leftrightarrow$   &  \verb!\Leftrightarrow! \\
$\dot a$        & \verb!\dot a!  	&
$\hat a$        & \verb!\hat a!  	&
$\bar a$        & \verb!\bar a!  	&
%$\tilde a$      & \verb!\tilde a!  	\\
$\overline a$      & \verb!\overline a!  	\\
%$\overline {(a+b)}$      & \verb!\overline {(a+b)}!  	\\
$\alpha$        &  \verb!\alpha!  	&
$\beta$         &  \verb!\beta!  	&
$\gamma$        &  \verb!\gamma!  	&
$\delta$        &  \verb!\delta!  	\\
$\epsilon$      &  \verb!\epsilon!  &
$\zeta$         &  \verb!\zeta!  	&
$\eta$          &  \verb!\eta!  	&
$\varepsilon$   &  \verb!\varepsilon!  \\
$\theta$        &  \verb!\theta!  	&
$\iota$         &  \verb!\iota!  	&
$\kappa$        &  \verb!\kappa!  	&
$\vartheta$     &  \verb!\vartheta! \\
$\lambda$       &  \verb!\lambda!  	&
$\mu$           &  \verb!\mu!  		&
$\nu$           &  \verb!\nu!  		&	
$\xi$           &  \verb!\xi!  		\\
$\pi$           &  \verb!\pi!  		&	
$\rho$          &  \verb!\rho!  	&
$\sigma$        &  \verb!\sigma!  	&
$\tau$          &  \verb!\tau!  	\\
$\upsilon$      &  \verb!\upsilon!  &
$\phi$          &  \verb!\phi!  	&
$\chi$          &  \verb!\chi!  	&
$\psi$          &  \verb!\psi!  	\\
$\omega$        &  \verb!\omega!  	&
$\Gamma$        &  \verb!\Gamma!  	&
$\Delta$        &  \verb!\Delta!  	&
$\Theta$        &  \verb!\Theta!  	\\
$\Lambda$       &  \verb!\Lambda!  	&
$\Xi$           &  \verb!\Xi!  		&
$\Pi$           &  \verb!\Pi!  		&
$\Sigma$        &  \verb!\Sigma!  	\\
$\Upsilon$      &  \verb!\Upsilon!  &
$\Phi$          &  \verb!\Phi!  	&
$\Psi$          &  \verb!\Psi!  	&
$\Omega$        &  \verb!\Omega!  
\end{tabular}
\footnotesize









%-------------------------
%-------------------------
\section{Unidades de longitud}
% Cualquier longitud se puede expresar en las siguientes unidades.\\
\begin{tabular}{@{}l@{\hspace{1em}}l@{\hspace{2em}}}
\textbf{mm} & milímetro $\approx 1/25$ pulgada $\approx 2,845$ pt\\
\textbf{cm} & centímetro $=10$ mm $\approx 28,45$ pt \\
\textbf{in} & pulgada $=25,4$ mm $\approx 72,27$ pt \\
\textbf{pt} & punto $\approx 1/72$ pulgada $\approx 1/28$ cm \\
\textbf{em} & $\approx$ anchura de una `M' en la tipografía actual \\
\textbf{ex} & $\approx$ altura de una `x' en la tipografía actual\\[0.7mm]
\end{tabular}


Una long. se puede expresar como proporción de longitud: \\
P.ej. \verb!0.5\textwidth! (la mitad del ancho de texto)\\
\verb!\the\!\texttt{<var>}: Muestra valor de variable \texttt{<var>} (sólo válido para longitudes)\\






%-------------------------
\section{Objetos flotantes (figuras y tablas)}
\settowidth{\MyLen}{\texttt{.begin.equation..place.}}
\begin{tabular}{@{}p{\the\MyLen}%
                @{}p{\linewidth-\the\MyLen}@{}}
\verb!\begin{figure}[<pos>]!:    &  Añade una figura numerada \\
\verb!\begin{table}[<pos>]!:     &  Añade un cuadro (tabla) numerado \\
\end{tabular}
Donde \texttt{<pos>} es una cadena indicativa de la posición con:\\
\texttt{t}=top,  \texttt{b}=bottom, \texttt{h}=here, \texttt{p}=page, \texttt{!}=forzar aquí\\
\verb!\caption[!\emph{tít.corto}\verb!]{!\emph{título}\verb!}!: Título de objeto flotante (incluido en el entorno)\\
\ding{118}  Recuerda: Coloca las etiquetas para refs. justo tras el título\\
Las tablas y listados con título en la parte superior y las figuras en la inferior\\




%-------------------------
\subsection{Objetos apaisados en una página {\small (paquete \texttt{rotating})}}
\begin{tabular}{@{}p{\the\MyLen}%
		@{}p{\linewidth-\the\MyLen}@{}}
	\verb!\begin{sidewaysfigure}!    &:  Añade una figura apaisada\\
	\verb!\begin{sidewaystable}!     &:  Añade una tabla apaisada \\
\end{tabular}
 




%-------------------------
\subsection{Figuras {\small (paquete \texttt{graphicx})}}
\verb!\graphicspath{{!\emph{path}\verb!}}!: Directorios de figuras\\
\verb!\DeclareGraphicsExtensions{.pdf,.png,.jpg}!: Orden de sel. de figs.
\verb!\includegraphics[!\emph{opc.}\verb!]{!\emph{fich.}\verb!}!: Incluye figura en documento\\

\textbf{Opciones de \textbackslash \texttt{includegraphics}:}\\

\settowidth{\MyLen}{\texttt{height=.height=.} }
\begin{tabular}{@{}p{\the\MyLen}%
                @{}p{\linewidth-\the\MyLen}@{}}
\texttt{width=l}, \texttt{height=l}      &  Ancho,alto de figura (sólo especificar el mayor)\\
\texttt{scale=l}, \texttt{angle=l}       &  Escalado, giro (grados en sentido antihorario)\\
\texttt{page=n}       &  Especifica la pág. \texttt{n} en un archivo PDF multipágina\\[0.7mm]
\end{tabular}


\textbf{Formatos gráficos recomendados}\\
\texttt{latex}: sólo admite \texttt{.eps} (postscript encapsulado)\\
\texttt{pdflatex}:\\
\quad \texttt{\textbf{.pdf}} \ding{220} gráficos vectoriales,\\
\quad \texttt{\textbf{.jpg}} ($\geq$ 100 ppi) \ding{220} imágenes,$^*$\\
\quad \texttt{\textbf{.png}} ($\geq$75 ppi) \ding{220} capturas de pantalla$^*$\\
$^*$ Los fondos blancos y uniformes deben convertirse en transparentes.




%-------------------------
\subsection{Entorno \texttt{tabular} }
\verb!\begin{tabular}[pos]{cols}! \\
\verb!\begin{tabular*}{ancho}[pos]{cols}!: Con control de anchura\\
Con \texttt{\&} y \verb!\\! para separar columnas y filas\\[0.7mm]

\textbf{Especificación de columnas en entorno \texttt{tabular}}
\settowidth{\MyLen}{\texttt{p}\{\emph{width}\} \ }
\begin{tabular}{@{}p{\the\MyLen}@{}p{\linewidth-\the\MyLen}@{}}
\texttt{lcr}    &   Justificación en col. (\texttt{l}=\emph{left}, \texttt{c}=\emph{center}, \texttt{r}=\emph{right})  \\
%\texttt{c}    &   Columna centrada (\emph{center}).  \\
%\texttt{r}    &   Columna justificada a la derecha (\emph{right}). \\
\verb!p{!\emph{ancho}\verb!}!  &  Igual que %
                              \verb!\parbox[t]{!\emph{ancho}\verb!}! \\ 
\verb!@{!\emph{decl}\verb!}!   &  Inserta \emph{decl} en vez del 
                                   espacio entre columnas \\
\verb!|!      &   Inserta una línea vertical entre columnas\\[0.7mm] 
\end{tabular}

\textbf{Elementos de entorno \texttt{tabular}}
\settowidth{\MyLen}{\texttt{.cline.x-y..}}
\begin{tabular}{@{}p{\the\MyLen}@{}p{\linewidth-\the\MyLen}@{}}
\verb!\hline!           &  Línea horizontal entre filas  \\
\verb!\cline{!$x$\verb!-!$y$\verb!}!  &
                        Línea horizontal a través de las columnas $x$ e $y$ \\
\verb!\multicolumn{!\emph{n}\verb!}{!\emph{cols}\verb!}{!\emph{texto}\verb!}! \\
        &  Una celda atraviesa \emph{n} columnas, con la especificación \emph{cols}.\\[0.7mm] 
\end{tabular}

\verb!\setlength{\tabcolsep}{!$l$\verb!}!: fija el espacio entre columnas a $l$\\





%-------------------------
\subsection{Entorno \texttt{tabbing}}
\begin{tabular}{@{}l@{\hspace{1.5ex}}l@{\hspace{10ex}}l@{\hspace{1.5ex}}l@{}}
\verb!\=!  &   Fijar tabulador &
\verb!\>!  &   Ir a siguiente tabulador\\
\end{tabular}

Fijar tabuladores en líneas <<invisibles>> finalizándolas con \verb!\kill!\\
Usar \verb!\\! para separar líneas






%-------------------------
%-------------------------
\section{Cajas (boxes)}
\ding{118} Cualquier elemento se puede colocar en una caja (box) definida por: \emph{ancho (width), alto (height), profundidad (depth)} y posición del contenido (\texttt{l}=izda., \texttt{r}=dcha., \texttt{c}=central, \texttt{s}=expandida)\\[0.7mm]
 
\verb!\mbox{!\emph{texto}\verb!}!: Crea una caja con el texto (no sufre guionado)\\
\verb!\makebox[!\emph{ancho}\verb!][!\emph{pos.}\verb!]{!\emph{obj.}\verb!}!: Crea una caja con el contenido dado\\
\verb!\parbox[!\emph{ancho}\verb!][!\emph{pos.}\verb!]{!\emph{obj.}\verb!}!: Crea una caja tipo párrafo con el contenido\\
\verb!\fbox{!\emph{obj.}\verb!}!: Crea una caja de texto con borde\\
\verb!\framebox[!\emph{ancho}\verb!][!\emph{pos.}\verb!]{!\emph{obj.}\verb!}!: Crea una caja con borde y el contenido dado\\
\verb!\begin{minipage}[!\emph{pos.}\verb!][!\emph{alto}\verb!][!\emph{posRel.}\verb!]!: Crea una pequeña pág.\\
\verb!\fboxrule!: Grosor del borde de la caja (\emph{def.} 0.4pt)\\
\verb!\fboxsep!: Separación entre el borde y la caja (\emph{def.} 3.0pt)\\[0.7mm]




%-------------------------
\textbf{Dimensionado y rotación} {\small (paquetes \texttt{graphicx} y \texttt{rotating})}\\
\verb!\scalebox{!\emph{esc-H}\verb!}[!\emph{esc-V}\verb!]{!\emph{obj.}\verb!}!: Escala el contenido de una caja\\
\verb!\reflectbox{!\emph{obj.}\verb!}!: Crea una caja con el objeto reflejado\\
\verb!\resizebox{!\emph{ancho}\verb!}{!\emph{alto}\verb!}{!\emph{obj.}\verb!}!: Reajusta el tamaño de la caja\\
\verb!\resizebox*{!\emph{ancho}\verb!}{!\emph{altoTot}\verb!}{!\emph{obj.}\verb!}!: Reajusta el tamaño de la caja\\
\verb!\rotatebox[!\emph{opc.}\verb!]{!\emph{áng.}\verb!}{!\emph{obj.}\verb!}!: Rota una caja el ángulo indicado\\
\verb!\begin{sideways}!: Rotación de 90$^\circ$ (paquete \texttt{rotating})\\





%-------------------------
\section{Bibliografía}
\verb!\begin{thebibliography}{99}!: Bibliografía (con ancho máx. de [\emph{cita}])\\
\verb!\bibitem{!\emph{clave}\verb!}!: Entrada de la bibliografía\\
\verb!\cite[!\emph{txt}\verb!]{!\emph{clave1, clave2,...}\verb!}!: Añade [\emph{cita}] en el texto (opc. [\emph{txt}, \emph{cita}])\\
\verb!\nocite{!\emph{clave}\verb!}!: Añade la obra a la bibliografía, sin [\emph{cita}] en texto\\
\verb!\nocite{*}!: Todas las obras en la bibliografía aunque no estén citadas\\





%-------------------------
\subsection{Bibliografía con \texttt{bibtex}}

\verb!\bibliographystyle{!\emph{estilo}\verb!}!: Define el estilo de la bibliografía\\
\verb!\bibliography{!\emph{<fichero>.bib}\verb!}!: Incluye las fuentes citadas en el documento\\
\ding{118} Estilos usuales: \texttt{plain}, \texttt{acm}, \texttt{ieeetr}, \texttt{alpha}, \texttt{abbrv}, \texttt{unsrt}\\




%-------------------------
\subsection{Bibliografía con \texttt{biblatex}}

\verb!\usepackage[backend=biber,sortcites]{biblatex}!\\
Opciones: \verb|defernumbers,style,autolang,language|\\ 
Estilos: \verb|ieee,iso-numeric,apa,mla-new,trad-<bibtex>|\\
Usar junto a:\verb!\usepackage[autostyle]{csquotes}!\\
\verb!\addbibresource{<fichero>.bib}! (poner extensión)
\\[0.7mm]
Citas: \verb!\cite[ver][#p]{id.}! \ding{220} [ver \#id, pág. \#p]\\
\verb|\cite<tt>|: siendo \texttt{tt} title, url, year\\
\verb|\footcite,\parencite,\textcite|\\
\verb|\printbibliography[title=Título|: Imprime bibliografía\\
Opciones (impresión selectiva): \texttt{keyword}, \texttt{notkeyword}, \texttt{type}, \texttt{nottype}\\







%-------------------------
\section{Personalización de \LaTeX}
%En las definiciones de comandos y entornos \emph{args.} representa el número de argumentos y \emph{def.} la definición mediante instrucciones \LaTeX.\\[0.7mm]

\verb!\newcommand{!\emph{cmd.}\verb!}[!\emph{args.}\verb!]{!\emph{def.}\verb!}!: Define el nuevo comando \emph{cmd}\\
\verb!\renewcommand{!\emph{cmd.}\verb!}[!\emph{args.}\verb!]{!\emph{def.}\verb!}!: Redefine un comando \emph{cmd.} ya definido\\[0.7mm]

\ding{118} Algunos nombres que pueden redefinirse: \verb!\tablename!, \verb!\figurename!, \verb!\abstractname!, \verb!\contentsname!, \verb!\listfigurename!, \verb!\listtablename!, \verb!\bibname! (clases \texttt{report} y \texttt{book}), \verb!\refname! (clase \texttt{article})\\
\ding{118} Para añadir un título en el índice señalado al nivel deseado:\\
\verb|\addcontentsline{<índice>}{<nivel>}{<título>}|\\[0.7mm]
 
\verb!\newenvironment{!\emph{env.}\verb!}[!\emph{args.}\verb!]{!\emph{antes}\verb!}{!\emph{después}\verb!}!: Define el nuevo entorno \emph{env}\\
\verb!\renewenvironment!: Ídem anterior para \emph{env.} ya definido\\
\verb!\newlength{!\emph{var}\verb!}!: Define una nueva variable de longitud \emph{var}\\
\verb!\setlength{!\emph{var}\verb!}{!$l$\verb!}!: Asigna $l$ a una \emph{var}. \emph{long} \\[0.7mm]
 

\rule{0.3\linewidth}{0.25pt}
\scriptsize

\copyright{}  \href{https://www.esi.uclm.es/www/jsalido}{J\reflectbox{e}sús S\reflectbox{a}lido} 2011 (basado en una idea original de Winston Chang, 2006)\\
%\copyright{} Jesús Salido (2011) preparado para curso \LaTeX-UCLM (basado en una idea original de Winston Chang, 2006)
% Historial de versiones
\verb!$Revision: 4.01$, $Fecha: 23/07/2019$!
%\verb!$Revision: 4.0$, $Fecha: 15/02/2018$!
%\verb!$Revision: 3.1$, $Fecha: 09/02/2015$!
%\verb!$Revision: 3.0$, $Fecha: 27/02/2014$!
%\verb!$Revision: 2.2$, $Fecha: 27/02/2012$!
%\verb!$Revision: 2.0$, $Fecha: 3/02/2012$!
%\verb!$Revision: 1.0$, $Fecha: Marzo 2011$!

\end{multicols}
\end{document}


